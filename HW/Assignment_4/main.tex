%%%%%%%%%%%%%%%%%%%%%%%%%%%%%%%%%%%%%%%%%%%%%%%%%%%%%%%%%%%%%%%
% Welcome to the MAT320 Homework template on Overleaf -- just edit your
% LaTeX on the left, and we'll compile it for you on the right.
%%%%%%%%%%%%%%%%%%%%%%%%%%%%%%%%%%%%%%%%%%%%%%%%%%%%%%%%%%%%%%%
% --------------------------------------------------------------
% Based on a homework template by Dana Ernst.
% --------------------------------------------------------------
% This is all preamble stuff that you don't have to worry about.
% Head down to where it says "Start here"
% --------------------------------------------------------------

\documentclass[12pt]{article}

\usepackage[margin=1in]{geometry} 
\usepackage{amsmath,amsthm,amssymb}

\newcommand{\N}{\mathbb{N}}
\newcommand{\Z}{\mathbb{Z}}

\newenvironment{ex}[2][Exercise]{\begin{trivlist}
\item[\hskip \labelsep {\bfseries #1}\hskip \labelsep {\bfseries #2.}]}{\end{trivlist}}

\newenvironment{sol}[1][Solution]{\begin{trivlist}
\item[\hskip \labelsep {\bfseries #1:}]}{\end{trivlist}}


\begin{document}

% --------------------------------------------------------------
%                         Start here
% --------------------------------------------------------------

\noindent Janeet Bajracharya \hfill {\Large MATH309: Homework 3} \hfill \today

\begin{ex}{4.a} 
What is the mode of a normal distribution with parameters $\mu$ and $\sigma$.
\end{ex}

\begin{sol}\
For a normal distribution with parameters $\mu$ and $\sigma$:\\\\
$\mu$ = Median = Mode
\\\\
Therefore, Mode = $\mu$.
\end{sol}\
\newpage
\begin{ex}{4.b}
Does the uniform distribution with parameters A and B have a single mode? Why or why not?
\end{ex}
\begin{sol}
Uniform Distributions $f_X(x;A,B) = \frac{1}{B-A}$ is constant for all x. Hence the $f_X(x;A,B)_{MAX}$ is all $x$ such that $A\leq x \leq B$. Hence, such a distribution cannot have a Singular Mode.
\end{sol}
\newpage
\begin{ex}{4.c}
What is the mode of an exponential distribution with parameter $\lambda$? (Draw
a picture)
\end{ex}
\begin{sol}\
\[
f_X(x) = \lambda e^{-\lambda x}  {x\geq0}
\]
Since,
\[
0 < e^{-\lambda x} \leq 1
\]
\[
\text{$f_X(x)$ is Maximized when $e^{-\lambda x} = 1$ which is at $x=0$.}
\]\\
Therefore,
\[
f_X(0) = \lambda = f_X(x)_{MAX}
\]
\\
\[
\text{The Mode is at $x = 0$.}
\]

\includegraphics[scale]{image}

\end{sol}
\newpage
\begin{ex}{4.d} 
If X has a gamma distribution with parameters $\alpha$ and $\beta$, and $\alpha$ $>$ 1, find
the mode
\end{ex}
\begin{sol}\
\[
f_X(x) = \dfrac{\beta^\alpha}{\gamma(\alpha)} * x^{\alpha-1} * e^{-\beta x}
\]

\[
f_X(x) = \frac{\beta^\alpha}{\Gamma(\alpha)} x^{\alpha-1} \exp(-\beta x)
\]

Since we are trying to maximize the function for some x we have to take a derivative of the function to find the $x$ where the Derivative is zero. Hence, we omit the constants ie. the terms without x's from the equation as they are be removed through derivatives either ways. 

\[
\log(f_X(x)) = (\alpha - 1) \log(x) - \beta x + \text{constants}
\]
Taking the derivative of the natural log of the function.\\
\[
\frac{d}{dx}\left(\log(f_X(x))\right) = \frac{\alpha - 1}{x} - \beta
\]
\\\\
Set the derivative to be 0 for maximum point since its a PDF. 
\[
0 = \frac{\alpha-1}{x} - \beta
\]
Therefore, The Mode is at: 
\[
x = \frac{\alpha -1}{\beta}
\]
\end{sol}
\newpage
\begin{ex}{4.e}
What is the mode of a chi-squared distribution having $\nu$ degrees of freedom ? 
\end{ex}
\begin{sol}
\[
f_X(x) = \frac{2^{\nu/2} x^{\frac{\nu}{2} -1} e^{-\frac{x}{2}}}{\Gamma(\nu/2)}
\]

Taking the natural log of the equation and then taking the derivative with respect to x. Set the derivative to 0 to find the maximum since this is a continuous PDF and omit constants as the derivative would evaluate to zero. 
\[
\log{f_X(x)} = ({\frac{\nu}{2} -1})x - \frac{x}{2} + \text{constants}w
\]
\[
\frac{d}{dx}(log{f_X(x)}) = ({\frac{\nu}{2} -1})\frac{d}{dx} {\log{x}} - \frac{d}{dx}{\frac{x}{2}}
\]

\[
0 = {(\frac{\nu}{2} -1)} * \frac{1}{x} - \frac{1}{2}
\]

\[
\frac{1}{2} = \frac{\nu -2}{2x}
\]
\\ 
\[
x = \nu -2
\]
\\
The Mode exists at $x = \nu -2$ where $\nu \geq 2$. If $\nu \leq 2$ then the Mode is 0.
\end{sol}

% --------------------------------------------------------------
%     You don't have to mess with anything below this line.
% --------------------------------------------------------------

\end{document}
